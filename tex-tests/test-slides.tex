\documentclass[11pt,xcolor={svgnames},aspectratio=169,usepdftitle=false]{beamer}

% Add space in lists
\let\toneitemize\itemize
\let\ttwoitemize\enditemize
\renewenvironment{itemize}{\toneitemize\addtolength{\itemsep}{0.7\baselineskip}}{\ttwoitemize}

\let\toneenumer\enumerate
\let\ttwoenumer\endenumerate
\renewenvironment{enumerate}{\toneenumer\addtolength{\itemsep}{0.7\baselineskip}}{\ttwoenumer}

% Alert color
\setbeamercolor{alerted text}{fg=DarkOrange}

% Itemize w bullets
\setbeamertemplate{itemize items}{$\circ$}

% Font size of title
\setbeamerfont{title}{size=\huge}

%===========================================================
% Color specifications
%===========================================================
\definecolor{GreyishBlue}{HTML}{087E8B} 	% Headings color
\definecolor{CitationsBlue}{HTML}{0E86D4}%{2B6684} 	% Color for citations
\definecolor{LinksPink}{HTML}{ff0054}       % Color for links

% Template for syntax highlighting
\definecolor{BeigeBackground}{HTML}{fdf6e3}
\definecolor{BlueKeyWord}{HTML}{268bd2}

\setbeamercolor{palette primary}{bg=GreyishBlue,fg=white}
\setbeamercolor{palette secondary}{bg=GreyishBlue,fg=white}
\setbeamercolor{palette tertiary}{bg=GreyishBlue,fg=white}
\setbeamercolor{palette quaternary}{bg=GreyishBlue,fg=white}
\setbeamercolor{structure}{fg=GreyishBlue} % itemize, enumerate, etc
\setbeamercolor{section in toc}{fg=GreyishBlue} % TOC sections
\setbeamercolor{background canvas}{bg=white}
\setbeamercolor{button}{bg=white, fg=GreyishBlue}

%===========================================================
% Language and font settings
%===========================================================
\usepackage[english,activeacute]{babel} % Language
\usefonttheme{professionalfonts}		% Avoid overwriting fonts

\usepackage[sfdefault,light]{FiraSans}
\usepackage[utf8]{inputenc}				% Special symbols
\usepackage[T1]{fontenc}				% T1 Encoding of font

%===========================================================
% Math font settings
%===========================================================
\usepackage{mathpazo}					% Math font
\usepackage{amsmath,amsfonts,amssymb,amsthm}	% Math symbols
\usepackage{dsfont}						% Math symbols like R for reals...

%===========================================================
% Numeration in environments
%===========================================================
\setbeamertemplate{theorems}[numbered]
\undef{\definition}
\newtheorem{definition}{Definition}
\setbeamertemplate{definitions}[numbered]
\setbeamertemplate{navigation symbols}{}
\setbeamertemplate{caption}[numbered]
\newtheorem{exercise}{Exercise}

\usepackage{appendixnumberbeamer}

%===========================================================
% References' colors and PDF properties
%===========================================================
\usepackage{hyperref}
\hypersetup
{
    pdfauthor={Rafael Serrano Quintero},
    pdfsubject={Introduction to Matlab},
    colorlinks = {true},
    linkcolor = {GreyishBlue},
    citecolor = {CitationsBlue},
    urlcolor = {LinksPink},
    filecolor = {LinksPink}
}
%===========================================================
% Additional packages
%===========================================================
\usepackage{graphicx}
\usepackage{tikz}
\usetikzlibrary{calc,fadings,arrows,shapes}
\usepackage{longtable}
\usepackage{appendix}
\usepackage{marvosym}
\usepackage{epstopdf}
\usepackage[round]{natbib}
% Change manually the color of the parenthesis
\bibpunct{\textcolor{CitationsBlue}{(}}{\textcolor{CitationsBlue}{)}}{,}{a}{}{;}

\usepackage[flushleft]{threeparttable}
\usepackage{booktabs}
\usepackage[super]{nth}
\usepackage{float}
\usepackage{caption}
\usepackage{subcaption}
\usepackage{multicol}

%===========================================================
% Stata Preamble for Tables
%===========================================================

\newcommand{\sym}[1]{\rlap{#1}}% Thanks to David Carlisle

\let\estinput=\input% define a new input command so that we can still flatten the document

\newcommand{\estwide}[3]{
		\vspace{.75ex}{
			\begin{tabular*}
			{\textwidth}{@{\hskip\tabcolsep\extracolsep\fill}l*{#2}{#3}}
			\toprule
			\estinput{#1}
			\bottomrule
			\addlinespace[.75ex]
			\end{tabular*}
			}
		}

\newcommand{\estauto}[3]{
		\vspace{.75ex}{
			\begin{tabular}{l*{#2}{#3}}
			\toprule
			\estinput{#1}
			\bottomrule
			\addlinespace[.75ex]
			\end{tabular}
			}
		}

% Allow line breaks with \\ in specialcells
	\newcommand{\specialcell}[2][c]{%
	\begin{tabular}[#1]{@{}c@{}}#2\end{tabular}}

%*****************************************************************
% Custom subcaptions
%*****************************************************************
% Note/Source/Text after Tables
\newcommand{\figtext}[1]{
	\vspace{-1.9ex}
	\captionsetup{justification=justified,font=footnotesize}
	\caption*{\hspace{6pt}\hangindent=1.5em #1}
	}
\newcommand{\fignote}[1]{\caption*{\footnotesize\emph{Note:~}~#1}}

\newcommand{\figsource}[1]{\figtext{\emph{Source:~}~#1}}

% Add significance note with \starnote
\newcommand{\starnote}{\figtext{* p < 0.1, ** p < 0.05, *** p < 0.01. Standard errors in parentheses.}}

%*****************************************************************
% siunitx
%*****************************************************************
\usepackage{siunitx} % centering in tables
	\sisetup{
		detect-mode,
		tight-spacing           = true,
		group-digits            = false ,
		input-signs             = ,
		input-symbols           = ( ) [ ] - + *,
		input-open-uncertainty  = ,
		input-close-uncertainty = ,
		table-align-text-post   = false
        }

% Code highlighting
\usepackage{fancyvrb}  %To reduce font size in verbatim environment
\usepackage{listings}

\lstloadlanguages{[LaTeX]TeX}%
\lstset{
    language=Matlab,
    basicstyle=\footnotesize\ttfamily,
    backgroundcolor=\color{BeigeBackground}, 
    breaklines=true,                 
    captionpos=b,                    
    commentstyle=\color{gray},   
    frame=none,
    keywordstyle=[1]\color{BlueKeyWord}\bfseries,
    keywordstyle=[2]\color{Purple},
    keywordstyle=[3]\color{Purple}\underbar,
    stringstyle=\color{Purple},   
    morekeywords={xlim,ylim,var,alpha,factorial,poissrnd,normpdf,normcdf,ones},
    morekeywords=[2]{all, on, off, interp},
    morecomment=[s]{\%\{}{\%\}},
    numbers=left,                    
    numbersep=3pt,                   
    numberstyle=\tiny\color{black},
    rulecolor=\color{black},        
    showspaces=false,               
    showstringspaces=false,          
    showtabs=false,                  
    stepnumber=1,                    
    tabsize=2         
}

%===========================================================
% DOCUMENT
%===========================================================

\begin{document}
\title[]{Nice Title for Nice Presentation}
\author{First Author\inst{1} \and
Second Author\inst{2} \and
Third Author\inst{3}}

\institute{\inst{1} Institution One \and \inst{2} Institution Two \and \inst{3} Institution Three}
\date{}

\AtBeginSection[]{\frame{\sectionpage}}
\AtBeginSubsection[]{\frame{\subsectionpage}}

\defbeamertemplate{section page}{mine}[1][]{%
  \begin{centering}
    {\usebeamerfont{section name}\usebeamercolor[fg]{section name}#1}
    \vskip1em\par
    \begin{beamercolorbox}[sep=12pt,center]{part title}
      \usebeamerfont{section title}\insertsection\par
    \end{beamercolorbox}
  \end{centering}
}

\defbeamertemplate{subsection page}{mine_sub}[1][]{%
  \begin{centering}
    {\usebeamerfont{subsection name}\usebeamercolor[fg]{subsection name}#1}
    \vskip1em\par
    \begin{beamercolorbox}[sep=12pt,center]{part title}
      \usebeamerfont{subsection title}\insertsubsection\par
    \end{beamercolorbox}
  \end{centering}
}

\setbeamertemplate{section page}[mine]
\setbeamertemplate{subsection page}[mine_sub]

\begin{frame}
	\titlepage
\end{frame}

\begin{frame}
	\frametitle{Title for Slide}

A titled slide.

Maybe some \alert{\textbf{alert instructions}}

\end{frame}

\section{Typesetting}

\begin{frame}[fragile]
	\frametitle{Typesetting}

\begin{itemize}
	\item The main fonts are set with \verb;FiraSans; and keep \verb;professionalfonts;.
	\item In this theme we have \emph{emphasized}, \alert{alerted}, and \textbf{bold} text.
	\item Citations and links are colored differently. \citet{acemoglu2008capitaldeepening} is a citation, while \href{https://github.com/rafserqui}{this is a link}.
\end{itemize}
\end{frame}

\section{Math}

\begin{frame}
	\frametitle{Math}
\begin{itemize}
	\item Inline equations such as $e^{i \pi} + 1 = 0$
	\item Centered equations can be unnumbered.
	\[
	\int f(x)dx = F(x)
	\]
	\item Or numbered and aligned
	\begin{align}
        \underset{\vec{K}}{\max} \phantom{\Omega} & \int_0^{\infty} U(C_t) e^{-it} dt \label{eqn:objective} \\
        \text{such that } & C_t = Y_t - \dot{K}_t = F(K_t,L_t,t) - \dot{K}_t \label{eqn:constraint}
    \end{align}
\end{itemize}
\end{frame}

\section{Blocks and Lists}

\begin{frame}
	\frametitle{Blocks}
\begin{block}{Block}
Regular block. Notice no visible box around the block. Keep it simple.
\end{block}

\begin{alertblock}{Alert Block}
This is an alert block. Again, no boxes around.
\end{alertblock}

\begin{exampleblock}{Example Block}
This is an example block. No boxes either.
\end{exampleblock}
\end{frame}

\begin{frame}
	\frametitle{Theorems}
\begin{theorem}
Suppose $g(x)$ is a given, continuous function defined on $[a,b]\subseteq\mathbb{R}$. If
	\[
	\int_a^b \eta(x)g(x) dx = 0
	\]
for every continuous function $\eta(x)$ defined on $[a,b]$ and satisfying $\eta(a) = \eta(b) = 0$ then, $g(x) = 0$ for $a\leq x\leq b$
\end{theorem}
\end{frame}

\begin{frame}{Lists}
	\begin{columns}[t, onlytextwidth]
		\column{0.33\textwidth}
			Items:
			\begin{itemize}
				\item Item 1
				\begin{itemize}
					\item Subitem 1.1
					\item Subitem 1.2
				\end{itemize}
				\item Item 2
				\item Item 3
			\end{itemize}
		
		\column{0.33\textwidth}
			Enumerations:
			\begin{enumerate}
				\item First
				\item Second
				\begin{enumerate}
					\item Sub-first
					\item Sub-second
				\end{enumerate}
				\item Third
			\end{enumerate}
		
		\column{0.33\textwidth}
			Descriptions:
			\begin{description}
				\item[First] Yes.
				\item[Second] No.
				\item[Third] Maybe?
			\end{description}
	\end{columns}
\end{frame}

\section{Figures and Tables}

\begin{frame}{Figures and Tables}
	\begin{columns}
		\column{0.4\textwidth}
			\begin{figure}
				\centering
				\includegraphics[width=\textwidth]{example-image-a}
				\caption{Figure Caption}
				\label{fig:example_fig}
			\end{figure}
			
		\column{0.6\textwidth}
		\begin{table}[htbp]
			\centering
			\caption{Table Caption}
			\label{tab:calibration}
			\scalebox{0.92}{
			\begin{tabular}{@{}rcc@{}}
				\toprule
					& Heading 1 & Heading 2 \\\midrule
				Row 1 & \(v_{11}\) & \(v_{12}\) \\
				Row 2 & \(v_{21}\) & \(v_{22}\) \\
				Row 3 & \(v_{31}\) & \(v_{32}\) \\ \bottomrule
			\end{tabular}
			}
		\end{table}
	\end{columns}
\end{frame}

\begin{frame}
	\frametitle{TikZ Figures}
\begin{columns}
	\begin{column}{0.5\textwidth}
		\begin{figure}
			\centering
			\begin{tikzpicture}[scale=0.5]
				\draw[thick,<->] (0,10) node[left]{$\delta k$, $sf(k)$}--(0,0)--(10,0) node[right]{$k$};
				\node [below left] at (0,0) {$0$};
				\draw(0,0)--(9,9) node[right]{$\delta k$};
				\draw(0,0) ..controls (1,5) and (5,6) .. (10,7) node[right]{$sf(k)$};
			\end{tikzpicture}
			\caption{A TikZ Figure Caption}
			\label{fig:tikz_figure}
		\end{figure}
	\end{column}
	\begin{column}{0.5\textwidth}
		\pgfdeclarelayer{background}
		\pgfsetlayers{background,main}
		\tikzstyle{vertex}=[circle,fill=black!25,minimum size=20pt,inner sep=0pt]
		\tikzstyle{selected vertex} = [vertex, fill=red!24]
		\tikzstyle{edge} = [draw,thick,-]
		\tikzstyle{weight} = [font=\small]
		\tikzstyle{selected edge} = [draw,line width=5pt,-,red!50]
		\tikzstyle{ignored edge} = [draw,line width=5pt,-,black!20]
		\begin{figure}
			\begin{tikzpicture}[scale=1.1, auto,swap]
				% Draw a 7,11 network
				% First we draw the vertices
				\foreach \pos/\name in {{(0,2)/a}, {(2,1)/b}, {(4,1)/c},
										{(0,0)/d}, {(3,0)/e}, {(2,-1)/f}, {(4,-1)/g}}
					\node[vertex] (\name) at \pos {$\name$};
				% Connect vertices with edges and draw weights
				\foreach \source/ \dest /\weight in {b/a/7, c/b/8,d/a/5,d/b/9,
													 e/b/7, e/c/5,e/d/15,
													 f/d/6,f/e/8,
													 g/e/9,g/f/11}
					\path[edge] (\source) -- node[weight] {$\weight$} (\dest);
				% Start animating the vertex and edge selection. 
				\foreach \vertex / \fr in {d/1,a/2,f/3,b/4,e/5,c/6,g/7}
					\path<\fr-> node[selected vertex] at (\vertex) {$\vertex$};
				% For convenience we use a background layer to highlight edges
				% This way we don't have to worry about the highlighting covering
				% weight labels. 
				\begin{pgfonlayer}{background}
					\pause
					\foreach \source / \dest in {d/a,d/f,a/b,b/e,e/c,e/g}
						\path<+->[selected edge] (\source.center) -- (\dest.center);
					\foreach \source / \dest / \fr in {d/b/4,d/e/5,e/f/5,b/c/6,f/g/7}
						\path<\fr->[ignored edge] (\source.center) -- (\dest.center);
				\end{pgfonlayer}
			\end{tikzpicture}
		\end{figure}
	\end{column}
\end{columns}
\end{frame}

\begin{frame}<beamer:0>
\bibliography{bibtex-test}
\bibliographystyle{ecta}
\end{frame}

\end{document}