\documentclass[11pt,xcolor={svgnames},aspectratio=169,usepdftitle=false,notheorems]{beamer}
\usepackage{clean_style_beamer}  % Loads my formatting 
\usepackage{colortbl}
\addbibresource{bibtex-test.bib}
\hypersetup
{
    pdfauthor={Rafael Serrano-Quintero},
    pdfsubject={Spatial Complementarities in Public Infrastructure and Structural Transformation in Brazil},
    colorlinks = {true},
    linkcolor = {LightBlue},
    citecolor = {LightGray},
    urlcolor = {LightBlue},
    filecolor = {LightBlue}
}
\usepackage{siunitx}
\sisetup{input-open-uncertainty  = ,
         input-close-uncertainty = ,
         table-space-text-pre    = (,
         table-space-text-post   = \sym{***},
         table-align-text-pre    = false,
         table-align-text-post   = false}
\def\sym#1{\ifmmode^{#1}\else\(^{#1}\)\fi}

%===========================================================
% DOCUMENT
%===========================================================

\begin{document}
\title[]{Nice Title for Nice Presentation}
\author{First Author\inst{1} \and
Second Author\inst{2} \and
Third Author\inst{3}}

\institute{\inst{1} Institution One \and \inst{2} Institution Two \and \inst{3} Institution Three}
\date{}

\begin{frame}
	\titlepage
\end{frame}

\begin{frame}
	\frametitle{Title for Slide}

A titled slide.

Maybe some \alert{\textbf{alert instructions}}

\end{frame}

\section{Typesetting}

\begin{frame}[fragile]
	\frametitle{Typesetting}

\begin{itemize}
	\item The main fonts are set with \verb;FiraSans; and keep \verb;professionalfonts;.
	\item In this theme we have \emph{emphasized}, \alert{alerted}, and \textbf{bold} text.
	\item Citations and links are colored differently. \textcite{acemoglu2008capitaldeepening} is a citation, while \href{https://github.com/rafserqui}{this is a link}.
\end{itemize}
\end{frame}

\section{Math}

\begin{frame}
	\frametitle{Math}
\begin{itemize}
	\item Inline equations such as $e^{i \pi} + 1 = 0$
	\item Centered equations can be unnumbered.
	\[
	\int f(x)dx = F(x)
	\]
	\item Or numbered and aligned
	\begin{align}
        \underset{\vec{K}}{\max} \phantom{\Omega} & \int_0^{\infty} U(C_t) e^{-it} dt \label{eqn:objective} \\
        \text{such that } & C_t = Y_t - \dot{K}_t = F(K_t,L_t,t) - \dot{K}_t \label{eqn:constraint}
    \end{align}
\end{itemize}
\end{frame}

\section{Blocks and Lists}

\begin{frame}
	\frametitle{Blocks}
\begin{block}{Block}
Regular block. Notice no visible box around the block. Keep it simple.
\end{block}

\begin{alertblock}{Alert Block}
This is an alert block. Again, no boxes around.
\end{alertblock}

\begin{exampleblock}{Example Block}
This is an example block. No boxes either.
\end{exampleblock}
\end{frame}

\begin{frame}
	\frametitle{Theorems}
\begin{theorem}
Suppose $g(x)$ is a given, continuous function defined on $[a,b]\subseteq\mathbb{R}$. If
	\[
	\int_a^b \eta(x)g(x) dx = 0
	\]
for every continuous function $\eta(x)$ defined on $[a,b]$ and satisfying $\eta(a) = \eta(b) = 0$ then, $g(x) = 0$ for $a\leq x\leq b$
\end{theorem}
\end{frame}

\begin{frame}{Lists}
	\begin{columns}[t, onlytextwidth]
		\column{0.33\textwidth}
			Items:
			\begin{itemize}
				\item Item 1
				\begin{itemize}
					\item Subitem 1.1
					\item Subitem 1.2
				\end{itemize}
				\item Item 2
				\item Item 3
			\end{itemize}
		
		\column{0.33\textwidth}
			Enumerations:
			\begin{enumerate}
				\item First
				\item Second
				\begin{enumerate}
					\item Sub-first
					\item Sub-second
				\end{enumerate}
				\item Third
			\end{enumerate}
		
		\column{0.33\textwidth}
			Descriptions:
			\begin{description}
				\item[First] Yes.
				\item[Second] No.
				\item[Third] Maybe?
			\end{description}
	\end{columns}
\end{frame}

\section{Figures and Tables}

\begin{frame}{Figures and Tables}
	\begin{columns}
		\column{0.4\textwidth}
			\begin{figure}
				\centering
				\includegraphics[width=\textwidth]{example-image-a}
				\caption{Figure Caption}
				\label{fig:example_fig}
			\end{figure}
			
		\column{0.6\textwidth}
		\begin{table}[htbp]
			\centering
			\caption{Table Caption}
			\label{tab:calibration}
			\scalebox{0.92}{
			\begin{tabular}{@{}rcc@{}}
				\toprule
					& Heading 1 & Heading 2 \\\midrule
				Row 1 & \(v_{11}\) & \(v_{12}\) \\
				Row 2 & \(v_{21}\) & \(v_{22}\) \\
				Row 3 & \(v_{31}\) & \(v_{32}\) \\ \bottomrule
			\end{tabular}
			}
		\end{table}
	\end{columns}
\end{frame}

\begin{frame}
	\frametitle{TikZ Figures}
\begin{columns}
	\begin{column}{0.5\textwidth}
		\begin{figure}
			\centering
			\begin{tikzpicture}[scale=0.5]
				\draw[thick,<->] (0,10) node[left]{$\delta k$, $sf(k)$}--(0,0)--(10,0) node[right]{$k$};
				\node [below left] at (0,0) {$0$};
				\draw(0,0)--(9,9) node[right]{$\delta k$};
				\draw(0,0) ..controls (1,5) and (5,6) .. (10,7) node[right]{$sf(k)$};
			\end{tikzpicture}
			\caption{A TikZ Figure Caption}
			\label{fig:tikz_figure}
		\end{figure}
	\end{column}
	\begin{column}{0.5\textwidth}
		\pgfdeclarelayer{background}
		\pgfsetlayers{background,main}
		\tikzstyle{vertex}=[circle,fill=black!25,minimum size=20pt,inner sep=0pt]
		\tikzstyle{selected vertex} = [vertex, fill=red!24]
		\tikzstyle{edge} = [draw,thick,-]
		\tikzstyle{weight} = [font=\small]
		\tikzstyle{selected edge} = [draw,line width=5pt,-,red!50]
		\tikzstyle{ignored edge} = [draw,line width=5pt,-,black!20]
		\begin{figure}
			\begin{tikzpicture}[scale=1.1, auto,swap]
				% Draw a 7,11 network
				% First we draw the vertices
				\foreach \pos/\name in {{(0,2)/a}, {(2,1)/b}, {(4,1)/c},
										{(0,0)/d}, {(3,0)/e}, {(2,-1)/f}, {(4,-1)/g}}
					\node[vertex] (\name) at \pos {$\name$};
				% Connect vertices with edges and draw weights
				\foreach \source/ \dest /\weight in {b/a/7, c/b/8,d/a/5,d/b/9,
													 e/b/7, e/c/5,e/d/15,
													 f/d/6,f/e/8,
													 g/e/9,g/f/11}
					\path[edge] (\source) -- node[weight] {$\weight$} (\dest);
				% Start animating the vertex and edge selection. 
				\foreach \vertex / \fr in {d/1,a/2,f/3,b/4,e/5,c/6,g/7}
					\path<\fr-> node[selected vertex] at (\vertex) {$\vertex$};
				% For convenience we use a background layer to highlight edges
				% This way we don't have to worry about the highlighting covering
				% weight labels. 
				\begin{pgfonlayer}{background}
					\pause
					\foreach \source / \dest in {d/a,d/f,a/b,b/e,e/c,e/g}
						\path<+->[selected edge] (\source.center) -- (\dest.center);
					\foreach \source / \dest / \fr in {d/b/4,d/e/5,e/f/5,b/c/6,f/g/7}
						\path<\fr->[ignored edge] (\source.center) -- (\dest.center);
				\end{pgfonlayer}
			\end{tikzpicture}
		\end{figure}
	\end{column}
\end{columns}
\end{frame}

\begin{frame}<beamer:0>
\bibliography{bibtex-test}
\bibliographystyle{ecta}
\end{frame}

\end{document}
